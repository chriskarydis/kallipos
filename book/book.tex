% Options for packages loaded elsewhere
\PassOptionsToPackage{unicode}{hyperref}
\PassOptionsToPackage{hyphens}{url}
%
\documentclass[
]{article}
\usepackage{amsmath,amssymb}
\usepackage{iftex}
\ifPDFTeX
  \usepackage[T1]{fontenc}
  \usepackage[utf8]{inputenc}
  \usepackage{textcomp} % provide euro and other symbols
\else % if luatex or xetex
  \usepackage{unicode-math} % this also loads fontspec
  \defaultfontfeatures{Scale=MatchLowercase}
  \defaultfontfeatures[\rmfamily]{Ligatures=TeX,Scale=1}
\fi
\usepackage{lmodern}
\ifPDFTeX\else
  % xetex/luatex font selection
\fi
% Use upquote if available, for straight quotes in verbatim environments
\IfFileExists{upquote.sty}{\usepackage{upquote}}{}
\IfFileExists{microtype.sty}{% use microtype if available
  \usepackage[]{microtype}
  \UseMicrotypeSet[protrusion]{basicmath} % disable protrusion for tt fonts
}{}
\makeatletter
\@ifundefined{KOMAClassName}{% if non-KOMA class
  \IfFileExists{parskip.sty}{%
    \usepackage{parskip}
  }{% else
    \setlength{\parindent}{0pt}
    \setlength{\parskip}{6pt plus 2pt minus 1pt}}
}{% if KOMA class
  \KOMAoptions{parskip=half}}
\makeatother
\usepackage{xcolor}
\setlength{\emergencystretch}{3em} % prevent overfull lines
\providecommand{\tightlist}{%
  \setlength{\itemsep}{0pt}\setlength{\parskip}{0pt}}
\setcounter{secnumdepth}{-\maxdimen} % remove section numbering
\newlength{\cslhangindent}
\setlength{\cslhangindent}{1.5em}
\newlength{\csllabelwidth}
\setlength{\csllabelwidth}{3em}
\newlength{\cslentryspacingunit} % times entry-spacing
\setlength{\cslentryspacingunit}{\parskip}
\newenvironment{CSLReferences}[2] % #1 hanging-ident, #2 entry spacing
 {% don't indent paragraphs
  \setlength{\parindent}{0pt}
  % turn on hanging indent if param 1 is 1
  \ifodd #1
  \let\oldpar\par
  \def\par{\hangindent=\cslhangindent\oldpar}
  \fi
  % set entry spacing
  \setlength{\parskip}{#2\cslentryspacingunit}
 }%
 {}
\usepackage{calc}
\newcommand{\CSLBlock}[1]{#1\hfill\break}
\newcommand{\CSLLeftMargin}[1]{\parbox[t]{\csllabelwidth}{#1}}
\newcommand{\CSLRightInline}[1]{\parbox[t]{\linewidth - \csllabelwidth}{#1}\break}
\newcommand{\CSLIndent}[1]{\hspace{\cslhangindent}#1}
\ifLuaTeX
  \usepackage{selnolig}  % disable illegal ligatures
\fi
\IfFileExists{bookmark.sty}{\usepackage{bookmark}}{\usepackage{hyperref}}
\IfFileExists{xurl.sty}{\usepackage{xurl}}{} % add URL line breaks if available
\urlstyle{same}
\hypersetup{
  hidelinks,
  pdfcreator={LaTeX via pandoc}}

\author{}
\date{}

\begin{document}

\hypertarget{refs}{}
\begin{CSLReferences}{0}{0}
\end{CSLReferences}

Borsuk, Amaranth. 2018. \emph{The Book}. MIT Press.

\hypertarget{refs}{}
\begin{CSLReferences}{0}{0}
\end{CSLReferences}

Card, Stuart K, Allen Newell, και Thomas P Moran. 1983. \emph{The
Psychology of Human-Computer Interaction}. L. Erlbaum Associates Inc.

Carlston, DG. 1985. {`Software People: inside the computer business'}.
New York: Prentice Hall.

Engelbart, Douglas C. 1962. \emph{Augmenting human intellect: A
conceptual framework}. SRI, Menlo Park, CA.

Freiberger, Paul, και Michael Swaine. 1984. \emph{Fire in the Valley:
the making of the personal computer}. McGraw-Hill, Inc.

Hertzfeld, Andy. 2004. \emph{Revolution in The Valley: The Insanely
Great Story of How the Mac Was Made}. " O'Reilly Media, Inc.".

Hiltzik, Michael. 1999. {`Dealers of Lightning: Xerox PARC and the
Dawning of the Computer Age'}.

Johnson, Jeff, Teresa L. Roberts, William Verplank, David Canfield
Smith, Charles H. Irby, Marian Beard, και Kevin Mackey. 1989. {`The
xerox star: A retrospective'}. \emph{Computer} 22 (9): 11--26.

Kay, Alan C. 1993. {`The early history of Smalltalk'}. \emph{ACM SIGPLAN
Notices} 28 (3): 69--95.

Lanier, Jaron. 2014. \emph{Who owns the future?} Simon; Schuster.

Licklider, Joseph Carl Robnett. 1960. {`Man-computer symbiosis'}.
\emph{IRE transactions on human factors in electronics}, τχ. 1: 4--11.

Papert, Seymour. 1980. \emph{Mindstorms: children, computers, and
powerful ideas}. Basic Books, Inc.

Raskin, Jef. 2000. \emph{The humane interface: new directions for
designing interactive systems}. Addison-Wesley Professional.

Rheingold, Howard. 2000. \emph{Tools for thought: The history and future
of mind-expanding technology}. MIT Press.

Smith, Douglas K, και Robert C Alexander. 1999. \emph{Fumbling the
future: How Xerox invented, then ignored, the first personal computer}.
iUniverse.

Waldrop, M Mitchell. 2001. \emph{The dream machine: JCR Licklider and
the revolution that made computing personal}. Viking Penguin.

Weizenbaum, Joseph. 1976. \emph{Computer power and human reason: From
judgment to calculation.} WH Freeman \& Co.

\hypertarget{refs}{}
\begin{CSLReferences}{0}{0}
\end{CSLReferences}

Buxton, Bill. 2010. \emph{Sketching user experiences: getting the design
right and the right design}. Morgan kaufmann.

Card, Stuart K, William K English, και Betty J Burr. 1978. {`Evaluation
of mouse, rate-controlled isometric joystick, step keys, and text keys
for text selection on a CRT'}. \emph{Ergonomics} 21 (8): 601--13.

Card, Stuart K, Thomas P Moran, και Allen Newell. 2018. \emph{The
psychology of human-computer interaction}. Crc Press.

Carroll, John M. 2000. \emph{Making use: scenario-based design of
human-computer interactions}. MIT press.

Moggridge, Bill. 2007. \emph{Designing interactions}. MIT press
Cambridge, MA.

Norman, Don. 2013. \emph{The design of everyday things: Revised and
expanded edition}. Basic books.

Papanek, Victor, και R Buckminster Fuller. 1972. \emph{Design for the
real world}. Thames; Hudson London.

Pering, Celine. 2002. {`Interaction design prototyping of communicator
devices: Towards meeting the hardware-software challenge'}.
\emph{interactions} 9 (6): 36--46.

Thackara, John. 2006. \emph{In the bubble: designing in a complex
world}. MIT press.

Winograd, Terry κ.ά. 1996. \emph{Bringing design to software}.
\(\{\)Addison-Wesley Professional\(\}\).

\hypertarget{refs}{}
\begin{CSLReferences}{0}{0}
\end{CSLReferences}

Garrett, Jesse James. 2010. \emph{Elements of user experience, the:
user-centered design for the web and beyond}. Pearson Education.

Hiltzik, Michael. 1999. {`Dealers of Lightning: Xerox PARC and the
Dawning of the Computer Age'}.

Igoe, Tom. 2007. \emph{Making things talk: Practical methods for
connecting physical objects}. " O'Reilly Media, Inc.".

McEwen, Adrian, και Hakim Cassimally. 2013. \emph{Designing the internet
of things}. John Wiley \& Sons.

Norman, Don. 2013. \emph{The design of everyday things: Revised and
expanded edition}. Basic books.

Norman, Donald A. 2004. \emph{Emotional design: Why we love (or hate)
everyday things}. Basic Civitas Books.

O'Sullivan, Dan, και Tom Igoe. 2004. \emph{Physical computing: sensing
and controlling the physical world with computers}. Course Technology
Press.

Shneiderman, Ben, και Pattie Maes. 1997. {`Direct manipulation vs.
interface agents'}. \emph{interactions} 4 (6): 42--61.

\hypertarget{refs}{}
\begin{CSLReferences}{0}{0}
\end{CSLReferences}

Andrew, Hunt, και Thomas David. 2000. {`The Pragmatic Programmer: From
Journeyman to Master'}. Addison Wesley Longman, Redwood City.

Banzi, Massimo, και Michael Shiloh. 2014. \emph{Getting started with
Arduino: the open source electronics prototyping platform}. Maker Media,
Inc.

Graham, Paul. 2004. \emph{Hackers \& painters: big ideas from the
computer age}. " O'Reilly Media, Inc.".

Grudin, Jonathan. 1990. {`The computer reaches out: the historical
continuity of interface design'}. Στο \emph{Proceedings of the SIGCHI
conference on Human factors in computing systems}, 261--68. ACM.

Ingalls, Daniel. 2020. {`The evolution of Smalltalk: from Smalltalk-72
through Squeak'}. \emph{Proceedings of the ACM on Programming Languages}
4 (HOPL): 1--101.

Markoff, John. 2005. \emph{What the dormouse said: How the sixties
counterculture shaped the personal computer industry}. Penguin.

McConnell, Steve. 2004. \emph{Code complete}. Pearson Education.

Noble, Joshua. 2009. \emph{Programming interactivity: a designer's guide
to Processing, Arduino, and OpenFrameworks}. " O'Reilly Media, Inc.".

Olsen, Dan. 2009. \emph{Building interactive systems: principles for
human-computer interaction}. Cengage Learning.

Reas, Casey, και Ben Fry. 2007. \emph{Processing: a programming handbook
for visual designers and artists}. 6812. Mit Press.

Thimbleby, H. 2007. \emph{press on: Principles of Interaction
Programming}. MIT Press, Cambridge.

Victor, Bret. 2012. {`Learnable programming: Designing a programming
system for understanding programs'}. 2012.
\url{http://worrydream.com/LearnableProgramming}.

\hypertarget{refs}{}
\begin{CSLReferences}{0}{0}
\end{CSLReferences}

Fogg, BJ. 2003. \emph{Persuasive Technology: Using Computers to Change
What We Think and Do}. Morgan Kaufmann.

Kaptelinin, Victor, και Mary Czerwinski. 2007. \emph{Beyond the desktop
metaphor: designing integrated digital work environments}. Τ. 1. The MIT
Press.

Krueger, M. W. 1991. \emph{Artificial Reality II}. Addison-Wesley.

Laurel, Brenda. 2013. \emph{Computers as theatre}. Addison-Wesley.

Levy, Steven. 1984. \emph{Hackers: Heroes of the computer revolution}.
Τ. 14. Anchor Press/Doubleday Garden City, NY.

Markoff, John. 2005. \emph{What the dormouse said: How the sixties
counterculture shaped the personal computer industry}. Penguin.

McCullough, Malcolm. 1998. \emph{Abstracting craft: The practiced
digital hand}. MIT press.

Norman, Don. 2014. \emph{Things that make us smart: Defending human
attributes in the age of the machine}. Diversion Books.

Norman, Donald A. 2004. \emph{Emotional design: Why we love (or hate)
everyday things}. Basic Civitas Books.

Reeves, Byron, και Clifford Ivar Nass. 1996. \emph{The media equation:
How people treat computers, television, and new media like real people
and places.} Cambridge university press.

Rheingold, Howard. 2000. \emph{The Virtual Community: Homesteading on
the Electronic Frontier}. MIT press.

Weizenbaum, Joseph. 1976. \emph{Computer power and human reason: From
judgment to calculation.} WH Freeman \& Co.

\hypertarget{refs}{}
\begin{CSLReferences}{0}{0}
\end{CSLReferences}

Baecker, Ronald M. 1993. \emph{Readings in groupware and
computer-supported cooperative work: Assisting human-human
collaboration}. Elsevier.

Barnet, Belinda. 2013. \emph{Memory machines: The evolution of
hypertext}. Anthem Press.

Berners-Lee, Tim. 1996. {`WWW: Past, present, and future'}.
\emph{Computer} 29 (10): 69--77.

Bolt, Richard A. 1978. {`Spatial data management system'}. MASSACHUSETTS
INST OF TECH CAMBRIDGE ARCHITECTURE MACHINE GROUP.

Bush, Vannevar κ.ά. 1945. {`As we may think'}. \emph{The atlantic
monthly} 176 (1): 101--8.

Garrett, Jesse James. 2010. \emph{Elements of user experience, the:
user-centered design for the web and beyond}. Pearson Education.

Licklider, Joseph Carl Robnett. 1960. {`Man-computer symbiosis'}.
\emph{IRE transactions on human factors in electronics}, τχ. 1: 4--11.

Malone, Thomas W, και Kevin Crowston. 1994. {`The interdisciplinary
study of coordination'}. \emph{ACM Computing Surveys (CSUR)} 26 (1):
87--119.

Nelson, Theodor H. 1974. {`Computer lib/Dream machines'}.

Nelson, Theodor H κ.ά. 2010. \emph{POSSIPLEX: movies, intellect,
creative control, my computer life and the fight for civilization: an
autobiography of Ted Nelson}. Mindful Press.

Packer, Randall, και Ken Jordan. 2002. \emph{Multimedia: from Wagner to
virtual reality}. WW Norton \& Company.

Sellen, Abigail J, και Richard HR Harper. 2003. \emph{The myth of the
paperless office}. MIT press.

Shiffman, Daniel. 2009. \emph{Learning Processing: a beginner's guide to
programming images, animation, and interaction}. Morgan Kaufmann.

\hypertarget{refs}{}
\begin{CSLReferences}{0}{0}
\end{CSLReferences}

Denning, Peter J, και Robert M Metcalfe. 1998. \emph{Beyond calculation:
The next fifty years of computing}. Springer Science \& Business Media.

Engelbart, Douglas. 1988. {`The augmented knowledge workshop'}. Στο
\emph{A history of personal workstations}, 185--248.

Freiberger, Paul, και Michael Swaine. 1984. \emph{Fire in the Valley:
the making of the personal computer}. McGraw-Hill, Inc.

Goldberg, Adele, επιμ. 1988. \emph{A History of Personal Workstations}.
New York, NY, USA: Association for Computing Machinery.

Hertzfeld, Andy. 2004. \emph{Revolution in The Valley
{{[}}Paperback{{]}}: The Insanely Great Story of How the Mac Was Made}.
" O'Reilly Media, Inc.".

Kay, Alan, και Adele Goldberg. 1977. {`Personal dynamic media'}.
\emph{Computer} 10 (3): 31--41.

Kernighan, Brian W. 2019. \emph{UNIX: A History and a Memoir}. Kindle
Direct Publishing.

Lanier, Jaron. 2017. \emph{Dawn of the new everything: Encounters with
reality and virtual reality}. Henry Holt; Company.

Laurel, Brenda. 2013. \emph{Computers as theatre}. Addison-Wesley.

Nelson, Ted. 2008. \emph{Geeks Bearing Gifts}. Mindful Pr.

Sellen, Abigail J, και Richard HR Harper. 2003. \emph{The myth of the
paperless office}. MIT press.

Waldrop, M Mitchell. 2001. \emph{The dream machine: JCR Licklider and
the revolution that made computing personal}. Viking Penguin.

\hypertarget{refs}{}
\begin{CSLReferences}{0}{0}
\end{CSLReferences}

Bardini, Thierry. 2000. \emph{Bootstrapping: Douglas Engelbart,
coevolution, and the origins of personal computing}. Stanford University
Press.

Bolter, Jay David, και Richard Grusin. 2000. \emph{Remediation:
Understanding new media}. mit Press.

Engelbart, Douglas C. 1962. \emph{Augmenting human intellect: A
conceptual framework}. SRI, Menlo Park, CA.

Gildall, Gary. 1993. \emph{Computer Connections: People, Places, and
Events in the Evolution of the Personal Computer Industry}. Unpublished.

Hiltzik, Michael. 1999. {`Dealers of Lightning: Xerox PARC and the
Dawning of the Computer Age'}.

Ihde, Don. 2012. \emph{Technics and praxis: A philosophy of technology}.
Τ. 24. Springer Science \& Business Media.

Ingalls, Daniel. 2020. {`The evolution of Smalltalk: from Smalltalk-72
through Squeak'}. \emph{Proceedings of the ACM on Programming Languages}
4 (HOPL): 1--101.

Kay, Alan C. 1993. {`The early history of Smalltalk'}. \emph{ACM SIGPLAN
Notices} 28 (3): 69--95.

Lakoff, George, και Mark Johnson. 2008. \emph{Metaphors we live by}.
University of Chicago press.

Lanier, Jaron. 2010. \emph{You are not a gadget: A manifesto}. Vintage.

Mumford, Lewis. 2010. \emph{Technics and civilization}. University of
Chicago Press.

Nelson, Theodor H. 2010. \emph{POSSIPLEX: movies, intellect, creative
control, my computer life and the fight for civilization: an
autobiography of Ted Nelson}. Mindful Press.

Raskin, Jef. 2000. \emph{The humane interface: new directions for
designing interactive systems}. Addison-Wesley Professional.

Roszak, Theodore. 1986. \emph{From Satori to Silicon Valley: San
Francisco and the American Counterculture}. Don't Call It Frisco Press.

Shasha, Dennis, και Cathy Lazere. 1998. \emph{Out of their minds: the
lives and discoveries of 15 great computer scientists}. Springer Science
\& Business Media.

Wirth, Niklaus, και Jürg Gutknecht. 1992. \emph{Project Oberon}.
Addison-Wesley Reading.

\hypertarget{ux3b5ux3b9ux3c3ux3b1ux3b3ux3c9ux3b3ux3ae}{%
\section{Εισαγωγή}\label{ux3b5ux3b9ux3c3ux3b1ux3b3ux3c9ux3b3ux3ae}}

\begin{quote}
Η μάθηση δεν είναι το αποτέλεσμα της διδασκαλίας, αλλά το αποτέλεσμα της
δραστηριότητας του μαθητή. John Holt
\end{quote}

Η κατασκευή της διάδρασης είναι μια σχετικά νέα γνωστική περιοχή, η
οποία δημιουργήθηκε από τη μεγάλη αποδοχή που γνώρισαν τα συστήματα
διάδρασης ανθρώπου και υπολογιστή σε ένα ευρύτατο φάσμα εφαρμογών της
καθημερινότητας και της εργασίας. Είναι τόσες πολλές οι ψηφιακές ανάγκες
των ανθρώπων σε διαφορετικές πτυχές της ζωής τους (π.χ.η ευζωία, η
ψυχαγωγία, η μάθηση, το εμπόριο, η εργασία κτλ.), ενώ ταυτόχρονα
δημιουργούνται συνέχεια νέες συσκευές αλλά και νέες συνδέσεις μεταξύ
τους, ώστε η κατασκευή της διάδρασης αναδεικνύεται οργανικά σε
πρωταγωνιστή στη σχεδίαση νέων ανθρώπινων και κοινωνικών δραστηριοτήτων.
Το βιβλίο αυτό βασίζεται στην άποψη ότι η κατασκευή της διάδρασης, εκτός
του ότι είναι μια σύνθεση πέρα από το άθροισμα των επιμέρους τμημάτων,
είναι κυρίως ένα νέο τεχνολογικό επίπεδο, το οποίο έχει τη δυνατότητα να
επαναπροσδιορίσει με καλό ή κακό τρόπο όλες τις ανθρώπινες και
κοινωνικές δραστηριότητες.

Συνήθως, όταν έχουμε μια νέα γνωστική περιοχή, οι επιστήμονες θα
προσπαθήσουν να την προσεγγίσουν μεθοδικά, σύμφωνα με τις τεχνικές που
έχουν δουλέψει σε παρόμοιες περιοχές στο παρελθόν. Για παράδειγμα, ο
προγραμματισμός αντιμετωπίζεται ως υποπερίπτωση της ευρύτερης περιοχής
των μηχανικών (π.χ., μηχανολόγοι μηχανικοί), αφού έχει να κάνει με την
κατασκευή και τη λειτουργία μιας μηχανής. Ταυτόχρονα, είναι λογικό η
διάδραση να αντιμετωπίζεται ως υποπερίπτωση της ευρύτερης περιοχής του
βιομηχανικού σχεδιασμού (όπως π.χ. η γραφιστική και η εργονομία). Στην
ειδική περίπτωση της κατασκευής της διάδρασης και με δεδομένο ότι
αναφερόμαστε σε μια σύνθετη περιοχή, διαφορετικού επιπέδου από τις
επιμέρους, δεν έχουμε την ευχέρεια να κάνουμε τις παραπάνω
απλουστεύσεις.

Οι συσκευές διάδρασης με τους υπολογιστές, και αντίστοιχα η κατασκευή
της διάδρασής τους, είναι έννοιες φευγαλέες τουλάχιστον για την περίοδο
από τη δεκαετία του 1970 μέχρι και τη δεκαετία του 2010, αφού η διάδραση
με τους υπολογιστές ξεκινάει από το τραπέζι και περνάει στα κινητά,
φορετά, και διάχυτα συστήματα. Tη δεκαετία του 1970, η τυπική μορφή του
προσωπικού υπολογιστή ήταν ο επιτραπέζιος υπολογιστής χωρίς γραφικό
περιβάλλον εργασίας, το οποίο υπήρξε αντικείμενο έρευνας στα εργαστήρια.
Τη δεκαετία του 1980, η γραφική επιφάνεια εργασίας έγινε εμπορικά
διαθέσιμη, ενώ, παράλληλα, το μεγαλύτερο μέρος του λογισμικού είχε
περάσει από τη γραμμή εντολών στα μενού και στις φόρμες, οπότε το
πληκτρολόγιο παρέμεινε η πιο δημοφιλής συσκευή εισόδου. Τη δεκαετία του
1990, η γραφική επιφάνεια εργασίας και το ποντίκι έγιναν ο κυρίαρχος
τρόπος διάδρασης με τον προσωπικό υπολογιστή, οπότε η συσκευή εισόδου
ποντίκι και η έμμεση διάδραση με αντικείμενα στην οθόνη μέσω του δείκτη
καθόρισε τα πιο δημοφιλή στυλ διάδρασης. Στα τέλη της δεκαετίας του
2000, ο κινητός υπολογιστής με οθόνη αφής έφερε στο προσκήνιο τις
χειρονομίες και την άμεση διάδραση στην οθόνη, ενώ τη δεκαετία του 2010,
ο υπολογιστής διαχέεται πέρα από το γραφείο, τόσο στο περιβάλλον όσο και
στο ανθρώπινο σώμα, δημιουργώντας έτσι ένα οικοσύστημα συσκευών και
εφαρμογών για τον χρήστη. Αντίστοιχα, η κατασκευή της διάδρασης
εξελίσσεται έτσι ώστε τα βασικά αρχέτυπα και εργαλεία να διευκολύνουν
τον χειρισμό των νέων συσκευών του χρήστη, όπως είναι το πληκτρολόγιο, η
οθόνη, το ποντίκι, η οθόνη αφής, κτλ.

Παράλληλα, και πάντα αλληλένδετα με την εξέλιξη του υλικού και της
φυσικής μορφής του υπολογιστή, έχουμε μια εξέλιξη του λογισμικού και του
στυλ διάδρασης με τον υπολογιστή, η οποία σχετίζεται περισσότερο με τις
εφαρμογές και τις διεργασίες του χρήστη. Οι πρώτες δημοφιλείς εφαρμογές
του προσωπικού υπολογιστή ήταν ο επεξεργαστής κειμένου και τα φύλλα
εργασίας, τα οποία αποτελούσαν το βασικό κίνητρο αγοράς κατά τις
δεκαετίες του 1970 και του 1980. Τη δεκαετία του 1990 είχαμε τη μεγάλη
υπόσχεση των εκπαιδευτικών και ψυχαγωγικών πολυμέσων, τα οποία τελικά
δεν έφτασαν στον τελικό χρήστη όπως αρχικά είχε σχεδιαστεί (μέσω της
καλωδιακής τηλεόρασης), αλλά περισσότερο μέσω του οπτικού δίσκου, των
κονσολών για βιντεο-παιχνίδια, και του διαδικτύου. Από το τέλος της
δεκαετίας του 2000, έχουμε την επικράτηση των κοινωνικών μέσων δικτύωσης
ως κυρίαρχο στυλ διάδρασης με τον υπολογιστή. Πλέον, όλες οι εφαρμογές,
ανεξάρτητα από το αν έχουν στόχο την παραγωγικότητα, την εκπαίδευση, την
ψυχαγωγία, τις εμπορικές συναλλαγές ή την πληροφόρηση, βασίζονται ή
τουλάχιστον έχουν μια διάσταση κοινωνικού δικτύου. Αντίστοιχα, η
κατασκευή της διάδρασης εξελίσσεται, έτσι ώστε τα βασικά αρχέτυπα και
εργαλεία να διευκολύνουν τον χειρισμό των οντοτήτων του χρήστη, όπως
είναι τα τοπικά αρχεία, τα πολυμέσα, τα υπερμέσα, το κοινωνικό δίκτυο,
κτλ.

Η κατασκευή της διάδρασης ανθρώπου και υπολογιστή, όπως είδαμε συνοπτικά
παραπάνω, έχει παραμείνει για πολύ καιρό μια φευγαλέα περιοχή, επειδή σε
κάθε χρονική περίοδο έχουμε διαφορετικές τεχνολογικές μορφές υπολογιστών
(π.χ. επιτραπέζιος, κινητός, φορετός, διάχυτος) διεπαφών με τους χρήστες
(π.χ. η γραμμή εντολών, το γραφικό περιβάλλον, οι χειρονομίες, η φυσική
γλώσσα) και εφαρμογών (π.χ. η προσομοίωση, το γραφείο, η πλοήγηση, η
φωτογραφία). Για παράδειγμα, ένας χρήστης υπολογιστών, ο οποίος έλαβε τη
βασική, τη δευτεροβάθμια, και την τριτοβάθμια εκπαίδευση τη δεκαετία του
1970, ή το πολύ μέχρι τα μισά της δεκαετίας του 1980, είναι πολύ πιθανό
να έχει μεγάλη εξοικείωση με τη γραμμή εντολών και τους επιτραπέζιους
υπολογιστές, αφού αυτή ήταν η βασική μορφή στα χρόνια της εκπαίδευσής
του. Αντίθετα, ένας χρήστης που έλαβε την εκπαίδευσή του μετά το 2000
και κατά τη δεκαετία του 2010, είναι πολύ πιθανό να μην έχει καθόλου
προσωπικό επιτραπέζιο υπολογιστή, αφού οι βασικές διεργασίες του χρήστη
αυτήν τη χρονική περίοδο (π.χ. αναζήτηση στον παγκόσμιο ιστό, κοινωνική
δικτύωση, ψηφιακό περιεχόμενο κτλ.) μπορούν να γίνουν εξίσου καλά, αν
όχι καλύτερα, με έναν κινητό υπολογιστή με διεπαφή χειρονομίας, η οποία
δεν απαιτεί σχεδόν καμία ανάπτυξη νέων δεξιοτήτων. Η αποδοχή και η
επικράτηση της έννοιας της ευχρηστίας, περισσότερο ως οικειότητας με τις
πρώτες εμπειρίες μας έχει αυξήσει μεν την προσβασιμότητα στην
πληροφορία, αλλά, ταυτόχρονα, έχει μειώσει τη διαφάνεια των τεχνολογιών
διάδρασης, καθώς και τις δεξιότητες που απαιτούνται για την κατασκευή
της διάδρασης.

Βλέπουμε, λοιπόν, ότι, στην πράξη, τόσο ο υπολογιστικός όσο και ο
ψηφιακός αλφαβητισμός είναι έννοιες περισσότερο σχετικές με τη
δημογραφία και την ημερομηνία γέννησης, παρά με μια διαχρονική αξία. Για
παράδειγμα, ο όρος υπολογιστής για πολλές δεκαετίες πριν τη δημιουργία
των πρώτων ηλεκτρονικών και ψηφιακών υπολογιστών αναφερόταν στον άνθρωπο
που έκανε μαθηματικούς υπολογισμούς για να φτιάξει τριγωνομετρικούς και
λογαριθμικούς πίνακες. Για αυτόν τον λόγο, το περιεχόμενο του βιβλίου
σκόπιμα αποφεύγει τις πιο νέες εξελίξεις και τα νέα προϊόντα, έτσι ώστε
να είναι όσο γίνεται πιο διαχρονικό. Η έμφαση δίνεται σε παλαιότερα
συστήματα, όχι επειδή υπάρχει μια ρετρολαγνεία, αλλά επειδή υπάρχουν
διαχρονικές τάσεις, οι οποίες είναι παρούσες και σε σύγχρονα προϊόντα
και οι οποίες ενδέχεται να επηρεάσουν τα μελλοντικά. Η μελέτη
παλαιότερων συστημάτων δεν έχει απλά ιστορικό χαρακτήρα, αλλά σκοπεύει
να φωτίσει εκείνα τα τεχνολογικά και ανθρωπιστικά μοτίβα που
εμφανίζονται και σε σύγχρονα συστήματα και, πολύ πιθανόν, και σε
μελλοντικά.

Εκτός από την έμφαση στα σύγχρονα και επίκαιρα συστήματα, τα περισσότερα
βιβλία σε θέματα τεχνολογίας προσπαθούν να χωρέσουν όσο γίνεται
περισσότερο περιέχομενο στο τυπικό μέγεθος ενός τυπωμένου ή ηλεκτρονικού
βιβλίου. Σε αυτό το βιβλίο, ο στόχος ήταν να καλύψουμε όσο γίνεται
περισσότερα θέματα σε όσο γίνεται μικρότερο χώρο, άρα και σε λιγότερο
χρόνο για τον αναγνώστη. Επιπλέον, το ύφος της γραφής παραμένει
προφορικό και σκόπιμα αποφεύγει το εγκυκλοπαιδικό, αφού όλες οι
πληροφορίες είναι πλέον διαθέσιμες σε ηλεκτρονικά μέσα, καθώς και στα
κλαδικά βιβλία αναφοράς του τομέα. Ακόμη, το βιβλίο συνοδεύεται από
πολλές εικόνες συσκευών και λογισμικού διάδρασης με τον χρήστη. Οι
εικόνες αυτές σκόπιμα παρουσιάζονται σε ζευγάρια με σχετικά εκτενείς
λεζάντες στην ίδια σελίδα, έτσι ώστε να παρέχουν μια παράλληλη διεπαφή
ανάγνωσης, η οποία είναι σίγουρα πολύ οικεία στην εποχή της εικόνας.
Ακόμη περισσότερες εικόνες και πρόσθετους τρόπους οργάνωσής τους θα βρει
ο αναγνώστης στην ιστοσελίδα του βιβλίου, όπου υπάρχουν εικόνες σε
χρονολόγια και σε διαφάνειες. Με αυτόν τον τρόπο, το βιβλίο γίνεται
περισσότερο προσβάσιμο στον αναγνώστη, ενώ παράλληλα, λειτουργεί
συμπληρωματικά με άλλες προσπάθειες.

Αυτό το βιβλίο απευθύνεται σε όσους εμπλέκονται με οποιονδήποτε ρόλο στη
σχεδιάση και στην κατασκευή συστημάτων διάδρασης ανθρώπου και
υπολογιστή. Επομένως, είναι χρήσιμο τόσο σε επαγγελματίες όσο και σε
φοιτητές μαθημάτων πληροφορικής, μηχανικής και σχεδίασης, οι οποίοι
θέλουν να αποκτήσουν εισαγωγικές γνώσεις στη συγκεκριμένη θεματική
περιοχή ή θέλουν να τακτοποιήσουν σκόρπιες γνώσεις. Επιπλέον, με
δεδομένη την εξάπλωση των εργαλείων της πληροφορικής σε πολλούς
συγγενείς τεχνολογικούς και επιστημονικούς κλάδους, αλλά και σε ακόμη
περισσότερους κλάδους που ωφελούνται ή ακόμη και επηρεάζονται από τις
εφαρμογές της, το βιβλίο αυτό απευθύνεται σε όλους αυτούς που
συμμετέχουν σε μια ομάδα που καλείται να σχεδιάσει ή να βελτιώσει ένα
διαδραστικό σύστημα το οποίο εμπλέκεται σε μια ανθρώπινη δραστηριότητα,
ανεξάρτητα από τον ρόλο τους και ανεξάρτητα από τη βασική τους
δεξιότητα.

Υπάρχουν πολλά βιβλία και ακόμη περισσότερες ελεύθερες πηγές στο δίκτυο,
τα οποία είναι πλούσια σε περιεχόμενο και σε εγκυκλοπαιδικές γνώσεις,
και στα οποία αξίζει να ανατρέξουμε κάθε φορά που θα έχουμε ένα
συγκεκριμένο ερώτημα ή όταν θέλουμε να ενημερωθούμε σε βάθος. Η ανάγνωση
ενός βιβλίου είναι μεν αναγκαία συνθήκη, αλλά όχι και ικανή για να
μεταδώσει πρακτικές γνώσεις, ακόμη και όταν ο αναγνώστης μπορεί να
θυμάται το περιεχόμενο. Για αυτόν τον σκοπό, το βιβλίο συνοδεύεται με
συμπληρωματικό πολυμεσικό περιεχόμενο και κυρίως με τη δυνατότητα για
την προσθήκη περιεχομένου από τους αναγνώστες σε δικό τους αντίγραφο του
πηγαίου κώδικα. Η εποικοδομητική μελέτη του συμπληρωματικού περιεχόμενου
δίνει τη δυνατότητα στον αναγνώστη να μεταβεί σταδιακά στη δραστηριότητα
της σκέψης και της συγγραφής και, μέσα από αυτήν την προσπάθεια, να
κατανόησει καλύτερα όχι απλά το περιεχόμενο, αλλά και την ευρύτερη
γνωστική περιοχή. Η ουσιαστική όμως κατανόηση πρακτικών ζητημάτων, όπως
η κατασκευή της διάδρασης, απαιτεί και την πρακτική ενασχόληση με τα
αντίστοιχα ζητήματα, η οποία υλοποιείται μέσα από τις προτάσεις για
πρόσθετες δραστηριότητες κατασκευής διαδραστικών συστημάτων.

Στα επόμενα κεφάλαια αυτού του βιβλίου μελετάμε εκείνα τα θέματα τα
οποία, ανεξάρτητα από τις τεχνολογικές εξελίξεις των τελευταίων
δεκαετιών, παραμένουν διαχρονικά και επίκαιρα.

\hypertarget{ux3c0ux3c1ux3ccux3bbux3bfux3b3ux3bfux3c2}{%
\section{Πρόλογος}\label{ux3c0ux3c1ux3ccux3bbux3bfux3b3ux3bfux3c2}}

\begin{quote}
Τα πράγματα που πρέπει να κάνεις, τα μαθαίνεις κάνοντάς τα. Αριστοτέλης
\end{quote}

Ο σκοπός αυτού του βιβλίου είναι να δώσει μια σύντομη εισαγωγή στα
συστήματα διάδρασης ανθρώπου και υπολογιστή και, κυρίως, να ενθαρρύνει
έναν κριτικό διάλογο αναφορικά με τις ατομικές και συλλογικές επιλογές,
οι οποίες έχουν διαμορφώσει τα σύγχρονα συστήματα. Η μελέτη των
παλαιότερων συστημάτων έχει ιστορικό χαρακτήρα μόνο σε μια πρώτη
επιφανειακή ανάγνωση, γιατί ο βασικός σκοπός είναι να εντοπιστούν
εκείνες οι συνθήκες, οι δυνάμεις και τα υλικά, τα οποία θα επιτρέψουν
την κατασκευή νέων συστημάτων. Μέσα από την κριτική ανάλυση των
παλαιότερων συστημάτων προκύπτουν ερμηνείες για τη μορφή τους. Επιπλέον,
η μελέτη των παλαιότερων συστημάτων αποκαλύπτει τις διαχρονικές αξίες
και τις βέλτιστες πρακτικές, οι οποίες μπορούν να οδηγήσουν σε καλύτερα
συστήματα διάδρασης, με τρόπο συστηματικό και με τεκμηριωμένες
παραδοχές.

Μια προσεκτική μελέτη των παραδοσιακών και των σύγχρονων συστημάτων
διάδρασης δείχνει ότι δημιουργήθηκαν σε ένα συγκεκριμένο τεχνολογικό και
πολιτισμικό πλαίσιο και ότι εξυπηρετούν συγκεκριμένα κίνητρα και
στόχους. Αυτή η διαπίστωση απελευθερώνει τον αναγνώστη, καθώς του
επιτρέπει να κατανοήσει όλα τα σύγχρονα συστήματα, απλά ως ένα
στιγμιότυπο μιας διαδρομής με πολλές εναλλακτικές, και όχι ως κάτι
αναπόφεκτο, ούτε καν ως αναγκαίο βήμα, γι' αυτά που θα μπορούσαν να
δημιουργηθούν. Πέρα από μια κριτική ανάγνωση της τεχνολογικής εξέλιξης,
το κυρίαρχο αφήγημα που διατρέχει το σώμα του κειμένου, αλλά και το
συμπληρωματικό του περιεχόμενο, είναι η έμφαση σε εκείνες τις
διαχρονικές τεχνολογίες και τεχνικές που επιτρέπουν την κατασκευή νέων
εναλλακτικών συστημάτων διάδρασης για τις ανάγκες του σήμερα αλλά και
του αύριο.

Αυτό που παραμένει διαχρονικό δεν είναι τόσο κάποια δεδομένη γραφική
διεπαφή, όπως αυτή του κινητού ή του επιτραπέζιου συστήματος, αλλά
κυρίως εκείνη η σύνθεση υλικών και δυνάμεων όπως οι διαδραστικές αξίες,
οι μέθοδοι, τα αρχέτυπα, οι τεχνικές, και τα μοντέλα τα οποία
δημιούργησαν εκείνες τις διεπαφές. Με αυτόν τον τρόπο, ο αναγνώστης
μαθαίνει κυρίως να σκέφτεται για τις συνθήκες και για τον τρόπο που
κατασκευάστηκαν τα υπάρχοντα διαδραστικά συστήματα, έτσι ώστε να
μπορέσει στη συνέχεια, αφού κρίνει και ερμηνεύσει το παρελθόν, να
συνθέσει τις δυνάμεις της σύγχρονης εποχής για την κατασκευή μελλοντικών
συστημάτων διάδρασης.

Οι παραπάνω παραδοχές επηρεάζουν το περιεχόμενο και τη μορφή του
βιβλίου. Το περιεχόμενο μοιάζει σαν να γράφτηκε τη δεκαετία του 2000,
έτσι ώστε σε δέκα χρόνια από σήμερα να έχει την ίδια αξία, αφού η
κατανόηση μας για εκείνα τα συστήματα θα έχει αλλάξει λίγο στο μεταξύ.
Επιπλέον, δίνεται έμφαση στα κλασικά συστήματα διάδρασης, γιατί με τη
βοήθεια της χρονικής απόστασης που έχουμε, μπορούμε να ξεχωρίσουμε πιο
εύκολα τις ιδιότητες εκείνες οι οποίες είναι διαχρονικές από εκείνες που
απλά εξυπηρετούσαν παλιά κίνητρα, ή που ταίριαζαν στο οργανωσιακό
πλαίσιο μιας εποχής ή ενός οργανισμού. Αντίστοιχα, και η μορφή του
βιβλίου ακολουθεί το περιεχόμενό του και συνοδεύεται από εικόνες
συστημάτων που θεωρούνται κλασικά, ανεξάρτητα από την αρχική αποδοχή
τους, με την προϋπόθεση ότι περιέχουν αξίες και ιδέες που τελικά
συναντάμε διαχρονικά.

Συνοπτικά, σε αυτό το βιβλίο γίνεται μια σύνθεση γνώσεων με στόχο την
έμπνευση του αναγνώστη, ο οποίος θα αναζητήσει περισσότερα έξω από αυτό,
και, γιατί όχι, θα μπει σε έναν διάλογο με τον συγγραφέα, στο αποθετήριο
ανοιχτού πηγαίου κειμένου του βιβλίου, το οποίο υπάρχει για αυτόν τον
σκοπό. Με αυτόν τον τρόπο, τόσο η συγγραφή όσο και η ανάγνωση αυτού του
βιβλίου, γίνονται με τη μορφή ενός κριτικού διαλόγου, όπως ακριβώς
δηλαδή και το πνεύμα που διατρέχει το βιβλίο απέναντι στις τεχνολογίες
των συστημάτων διάδρασης.

\end{document}
